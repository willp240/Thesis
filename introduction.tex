\chapter{Introduction} \label{sec:introductions}

Sakharov proposed in 1967\cite{sakharov} that the abundance of matter over anti-matter observed in the Universe could be explained by interactions which meet three conditions. These are that: baryon number is violated, such that a different number of baryons can be produced to anti-baryons; C (charge) and \CP (charge-parity) symmetries are violated, so that processes producing more baryons aren't compensated by processes producing more anti-baryons; and the interactions occur outside of thermal equilibrium, so the net amount of generated baryons isn't cancelled out by the time reverse of the process. 

\CP violation was first observed in the quark sector in 1964\cite{quarkcpv}, but it is not sufficient to alone explain the matter dominated Universe we observe. The PMNS mechanism for neutrino oscillations allows a \CP violating phase, \deltacp, which could be non-zero. It has been shown that \CP violation in leptons could produce the matter anti-matter imbalance through leptogenesis\cite{leptogenesis}. 

T2K \cite{PhysRevLett.121.171802} is a long baseline neutrino oscillation experiment, originally designed to make precision measurements of the oscillation parameters sin$^2\theta_{23}$, $\Delta m^2_{32}$, and sin$^2\theta_{13}$. However, one of the main focuses is now on measuring \deltacp to determine if \CP is violated in neutrino oscillations, and if so by how much. As the data sets for both neutrinos and anti-neutrinos increase, and with the prospect of future neutrino oscillation experiments such as Hyper-Kamiokande \cite{Abe:2018uyc} and DUNE \cite{acciarri2016longbaseline} on the horizon, we will soon be able to determine the value of \deltacp. However, with the increase in sample size the treatment of systematics becomes more important. To achieve the required sensitivity, improvements to the neutrino interaction model are required.

The T2K near detector prediction is fitted to data to reduce systematic uncertainties from the cross-section, flux, and detector models used to build the MC prediction. This process typically reduces the uncertainty on event rates at the far detector, Super-Kamiokande, from $\sim$15$\%$ to $\sim$5$\%$, without which T2K would not be able to make world leading measurements of oscillation parameters.

This thesis presents the results of the near detector fit for the 2020 oscillation analysis, as well as describing the updates to the models and fitting framework since the last analysis. Chapter \ref{sec:NeutrinoPhysics} outlines the history of neutrino physics, from initial discovery to current open questions. It also describes the theory of both neutrino interactions, particularly those present in the T2K detectors, and neutrino oscillations.

The T2K experimental setup is detailed in Chapter \ref{sec:T2K}, giving an overview of the beamline, the on-axis near detector INGRID, the off-axis ND280, and the far detector Super-Kamiokande. The simulation used to predict measurements at each of the detectors is also described.

The statistical treatment of data is discussed in Chapter \ref{sec:stats}. This work is a Bayesian analysis, which uses Markov Chain Monte Carlo (MCMC) for fitting systematics to data. The MCMC method, and Bayesian techniques for post-fit analysis and interpretation are introduced here.

This is followed by a description of the near detector fit and inputs for the 2020 oscillation analysis in Chapter \ref{sec:FitSetup}. Various improvements were made to the cross-section, flux, and detector models, as well as the fitting framework itself, since the previous analysis. The implementation of these, and the potential impact on the fit are discussed here.

The results of the fit are described in Chapter \ref{sec:2020Fit}. First the validations of the model and fitting framework are shown, before the final results and impact at the far detector are presented. The impact of these results on the sensitivity to oscillation parameter measurements is then studied.

In Chapter \ref{sec:hptpc}, the potential impact of a new technology for long baseline neutrino oscillation experiment near detectors is discussed. Finally, this thesis concludes with a summary of the results presented.

The three sets of systematics, flux, cross-section, and detector, were provided by working groups within T2K. The beam group produced a covariance matrix of the flux systematics. The Neutrino Interaction Working Group (NIWG) recommended the cross-section systematics. This was partly informed by the studies on the binding energy dial performed by the author of this thesis, presented in Section \ref{sec:eb}. The ND280 Selection, Systematics and Validations group recommended the detector systematics. I made correlated throws of these systematics to produce a covariance matrix, and performed the studies on detector binnings presented in Section \ref{sec:detbin}.

The implementation of these systematics into the analysis framework, the ND280 fits, validations, and postfit studies, were all my work. The joint near and far detector fits were run in parallel by several members of the MaCh3 group, including me. I produced the comparisons of the joint fit results for fixing the binding energy, and using uniform binning, presented in Section \ref{sec:joint2018}.

\newpage