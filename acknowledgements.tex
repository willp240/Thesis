\chapter*{Acknowledgements}

This analysis would not have been possible without the help and support of a number of people. Firstly, I would like to thank my supervisor, Dr. Asher Kaboth, for providing the opportunity to work on such an interesting project, and always being happy to give advice on everything from technical programming problems, to the bigger picture physics implications. And for the many Disney karaoke sessions.

I am also grateful to Dr. Clarence Wret and Dr. Patrick Dunne for the generous amount of time and support they have given me throughout my PhD, without which this analysis would not have progressed.

This analysis was validated against the other near detector fitting framework on T2K, the BANFF group, led by Dr. Mark Scott with Laura Munteanu and Joe Walsh. I am thankful for their many discussions on interpreting and comparing the results. I am also grateful to the analysis coordinators Prof. Kendall Mahn and Prof. Mark Hartz for their advice and guidance. I would also like to thank Ed Atkinsons, Kevin Wood, and Balint Radics, for their help on producing the joint Asimov fit presented in this thesis, and Luke Pickering for his help in implementing the binding energy dial. Additionally, thanks go to Artur Sztuc and Toby Nonnenmacher for many physics discussions, both in the UK and Japan.

Working on a large experiment such as T2K means there are too many people to thank in person, but I am grateful to the whole collaboration, and in particular everyone who has worked in the MaCh3 group, for all their contributions to the experiment and support they have provided me.

I would like to acknowledge that this work would not have been possible without the use of several computing facilities: Compute Canada's Cedar and Graham clusters, facilitated by Hiro Tanaka; Royal Holloway High Energy Physics Deparment's Faraday Cluster, ran by Simon George, Barry Green, Tony Fernandez and Tom Crane; and Imperial College's High Energy Physics Computing, maintained by Simon Fayer and Ray Beuselinck.

There are two aspects of my PhD which are not presented in this thesis. Firstly, working on the HPTPC prototype, particularly on the CERN beam test, which was an enjoyable experience. Special thanks go to Prof. Jocelyn Monroe, Prof. Morgan Wascko, Dr. Dom Brailsford, Dr. Alexander Deisting and Dr. Abbey Waldron for their guidance on this work. Secondly, I served as a DAQ expert at ND280 as part of my LTA. I appreciate the advice and support given by the DAQ group, most notably Dr. Helen O'Keefe, Prof. Alfons Weber, Prof. Giles Barr, and Dr. Trevor Stewart.

Finally, I would like to thank my brother, parents, step-parents, grandparents, great uncle and aunt, and whole family, for their support and encouragement throughout this process and always. And thank you to my partner, Laura, for everything. The banter has been just lovely.

\newpage