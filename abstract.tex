\chapter{}

\begin{center}
\textbf{Abstract} 
\newline
\end{center}

T2K is a long baseline neutrino oscillation experiment in Japan. It was designed to make precise measurements of the parameters governing neutrino oscillations, and has been taking data since 2011. A muon (anti-)neutrino beam is produced at the Japan Proton Accelerator Research Complex (J-PARC), and is aimed towards the Super-Kamiokande (SK) detector 295 km away. In this analysis, Markov Chain Monte Carlo is used to fit the Monte Carlo prediction to data from the near detector, ND280, which measures the neutrino flux and interaction cross-sections before oscillation. The flux and interaction models are parametrised using external data, T2K beam line monitoring measurements, and theoretical calculations to set the prior values and uncertainties. Several updates have been made to the data samples, cross-section model, and fitting framework used for the 2020 oscillation analysis to maximise the constraint on these systematics, and reduce the impact they have on oscillation results. The near detector fit is crucial for T2K to make world-leading oscillation parameter measurements. The analysis presented here used ND280 data from T2K runs 2-9, corresponding to 1.99$\times10^{21}$ protons on target (POT), to reduce the uncertainty on the SK prediction from 12-14$\%$ to 2-3$\%$. The reduction of systematic uncertainties in detectors for future long baseline neutrino experiments is also investigated. 


\newpage
